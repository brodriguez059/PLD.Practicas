%-----------
\definecolor{backcolour}{rgb}{0.95,0.95,0.92}
\definecolor{dkgreen}{rgb}{0,0.6,0}
\definecolor{gray}{rgb}{0.5,0.5,0.5}
\definecolor{mauve}{rgb}{0.58,0,0.82}

%-----------
\lstdefinestyle{codestyle}{
    basicstyle=\ttfamily\footnotesize,% the size of the fonts that are used for the code
    numbers=left,                   % where to put the line-numbers
    numberstyle=\tiny\color{gray},  % the style that is used for the line-numbers
    stepnumber=1,                   % the step between two line-numbers. If it's 1, each line will be numbered
    numbersep=5pt,                  % how far the line-numbers are from the code
    backgroundcolor=\color{backcolour},  % choose the background color. You must add \usepackage{color}
    showspaces=false,               % show spaces adding particular underscores
    showstringspaces=false,         % underline spaces within strings
    showtabs=false,                 % show tabs within strings adding particular underscores
    frame=single,                   % adds a frame around the code
    rulecolor=\color{black},        % if not set, the frame-color may be changed on line-breaks within not-black text (e.g. comments (green here))
    tabsize=2,                      % sets default tabsize to 2 spaces
    captionpos=b,                   % sets the caption-position to bottom
    breaklines=true,                % sets automatic line breaking
    breakatwhitespace=false,        % sets if automatic breaks should only happen at whitespace
    title=\lstname,                 % show the filename of files included with \lstinputlisting;
                                  % also try caption instead of title
    keywordstyle=\color{blue},      % keyword style
    commentstyle=\color{dkgreen},   % comment style
    stringstyle=\color{mauve},      % string literal style
    escapeinside={\%*}{*)},         % if you want to add a comment within your code
    morekeywords={*,...},            % if you want to add more keywords to the set
    keepspaces=true,
    numbersep=5pt,
    inputencoding = utf8,  % Input encoding
    extendedchars = true,  % Extended ASCII
    literate      =        % Support additional characters
      {á}{{\'a}}1  {é}{{\'e}}1  {í}{{\'i}}1 {ó}{{\'o}}1  {ú}{{\'u}}1
      {Á}{{\'A}}1  {É}{{\'E}}1  {Í}{{\'I}}1 {Ó}{{\'O}}1  {Ú}{{\'U}}1
      {à}{{\`a}}1  {è}{{\`e}}1  {ì}{{\`i}}1 {ò}{{\`o}}1  {ù}{{\`u}}1
      {À}{{\`A}}1  {È}{{\'E}}1  {Ì}{{\`I}}1 {Ò}{{\`O}}1  {Ù}{{\`U}}1
      {ä}{{\"a}}1  {ë}{{\"e}}1  {ï}{{\"i}}1 {ö}{{\"o}}1  {ü}{{\"u}}1
      {Ä}{{\"A}}1  {Ë}{{\"E}}1  {Ï}{{\"I}}1 {Ö}{{\"O}}1  {Ü}{{\"U}}1
      {â}{{\^a}}1  {ê}{{\^e}}1  {î}{{\^i}}1 {ô}{{\^o}}1  {û}{{\^u}}1
      {Â}{{\^A}}1  {Ê}{{\^E}}1  {Î}{{\^I}}1 {Ô}{{\^O}}1  {Û}{{\^U}}1
      {œ}{{\oe}}1  {Œ}{{\OE}}1  {æ}{{\ae}}1 {Æ}{{\AE}}1  {ß}{{\ss}}1
      {ç}{{\c c}}1 {Ç}{{\c C}}1 {ø}{{\o}}1  {å}{{\r a}}1 {Å}{{\r A}}1
      {ñ}{{\~n}}1  {Ñ}{{\~N}}1  {¿}{{?`}}1  {¡}{{!`}}1
      {°}{{\textdegree}}1 {º}{{\textordmasculine}}1 {ª}{{\textordfeminine}}1
      % ¿ and ¡ are not correctly displayed if inconsolata font is used
      % together with the lstlisting environment. Consider typing code in
      % external files and using \lstinputlisting to display them instead. 
}
\lstset{style=codestyle}
\lstset{inputencoding=utf8}

%-----------
\addto{\captionsspanish}{
\renewcommand*{\appendixpagename}{Ap\'{e}ndices}
\renewcommand*{\appendixtocname}{Ap\'{e}ndices}
}

%-----------
\hypersetup{
    colorlinks=true,
    linkcolor=black,
    filecolor=magenta,      
    urlcolor=blue,
    citecolor=blue,
    linktoc = all,
}
\urlstyle{same}

%-----------
\addbibresource{references.bib}

%-----------
\makeindex
\oddsidemargin 0.5cm
\evensidemargin -0.5cm
\textwidth 16cm
\textheight 21.8cm
\topmargin 0cm
\footskip 0.001cm
\pagestyle{fancy}
\renewcommand{\thepage}{\roman{page}}
\fancyhf{} 
\fancyhead[R]{\small\slshape\thepage}
\fancyhead[L]{Propuesta}


%-----------
\nocite{*}

%----------
\newcommand{\error}[1]{\textcolor{red}{#1}}
\newcommand{\warning}[1]{\textcolor{orange}{#1}}

%----------
\newcolumntype{M}{>{\centering\arraybackslash}m{\dimexpr.2\linewidth-2\tabcolsep}} %Columa especial
\newcolumntype{N}{>{\justifying\arraybackslash}m{\dimexpr.2\linewidth-2\tabcolsep}} %Columa especial
\newcolumntype{L}{>{\raggedright\arraybackslash}X}
\newcolumntype{J}{>{\justifying\arraybackslash}X}
%----------
\newcommand{\myrowcolour}{\rowcolor[gray]{0.925}}

\newcommand{\HRule}{\rule{\linewidth}{0.5mm}}

\frenchspacing
\widowpenalty=1000

\setlength{\parindent}{0cm} % anula indentacion de parrafos
\setlength{\parskip}{1.5ex plus 0.5ex minus 0.5ex}   % establece separacion entre parrafos a 8 puntos

\setlength\headheight{15pt}