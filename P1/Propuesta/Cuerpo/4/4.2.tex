\subsection{Competidores}

\subsubsection{Ultimate Chicken Horse}

La descripción de este videojuego se ha realizado ya en \ref{subsubsec:UCH}, así
que no se procederá a no repetir información.

\paragraph{Características destacadas}
\begin{itemize}
    \item Multijugador online y local hasta 4 jugadores.
    \item Dinámica de juego única que fomenta la competición.
    \item Complejo si se desea usar estrategia, simple si no.
    \item Distintos modos de juego.
\end{itemize}

\paragraph{Limitaciones:}
\begin{itemize}
    \item A pesar de ser un juego festivo, también es un juego que tiende a
    hacerse lento.
    \item Los niveles a escoger son estáticos.
    \item Los distintos personajes no tienen mecánica que los diferencien de los
    demás, lo único diferente es la estética.
    \item El juego no pertenece al género de acción, lo cual también se puede
    observar en la velocidad de movimiento de los personajes.
\end{itemize}

\subsubsection{Jackbox Party Packs}
% Nombre
% Desarrollador + Editor
% Año de publicación
\emph{Jackbox Party Packs}, desarrollados y publicados por \emph{Jackbox Games,
Inc.} desde el 2014, son una serie de videojuegos cada uno compuesto por una
serie de minijuegos distintos que se pueden jugar online sin necesidad de que
todos los participantes hayan comprado el juego.

% Página web
\url{https://www.jackboxgames.com/}

\paragraph{Características destacadas}
\begin{itemize}
    \item Tiene una multitud de minijuegos distintos.
    \item Se caracteriza por tener una velocidad de juego relativamente rápida y
    una partidas cortas.
    \item Sólo una persona debe comprar el juego para que varias personas puedan
    disfrutar del videojuego.
\end{itemize}

\paragraph{Limitaciones:}
\begin{itemize}
    \item El hecho de que el paquete consista de múltiples minijuegos hace que
    algunos no sean divertidos. Hay demasiada cantidad y poca calidad en
    general.
    \item El juego sólo está disponible en inglés.
    \item Hay que comprar distintas versiones del juego para poder acceder a
    distintos minijuegos.
\end{itemize}

\subsubsection{Pummel Party}
% Nombre % Desarrollador + Editor % Año de publicación
\emph{Pummel Party}, desarrollado y publicado por \emph{Rebuilt Games} en el
2018, es un juego festivo multijugador usando en la que se destruye a tus
compañeros con una variedad de objetos y se compite por puntos en una colección
de minijuegos distintos.

La página oficial es: % Página web
\url{http://www.rebuiltgames.com/}

\paragraph{Características destacadas}
\begin{itemize}
    \item Una gran cantidad de objetos para destruir a tus amigos.
    \item Múltiples minijuegos distintos que fomentan la competitividad.
    \item Traducido a múltiplos idiomas, incluido castellano.
\end{itemize}

\paragraph{Limitaciones:}
\begin{itemize}
    \item El juego es por turnos, por lo cual hay que esperar a que cada jugador
    tome una acción antes de poder avanzar.
    \item 
\end{itemize}
