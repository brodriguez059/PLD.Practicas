% ################################################################
% #######     CODIFICACIÓN DEL ARCHIVO              ##############
% ################################################################

% ************************* Input encoding ********************************
% \usepackage{ucs}
\usepackage[utf8]{inputenc} % Permite usar carácteres UTF8
% https://www.overleaf.com/learn/latex/Spanish

% ************************* Output encoding ********************************
\usepackage[T1]{fontenc} % Use 8-bit encoding that has 256 glyphs
% https://www.overleaf.com/learn/latex/Spanish
% \usepackage{courier}

% ************************* Font types ********************************
% times erabili beharrean
\usepackage{mathptmx}
\usepackage[scaled=.90]{helvet}

% ************************* Lenguaje a utilizar ********************************

\usepackage[spanish, es-tabla]{babel} %To extend the default capabilities of LaTeX, providing proper hyphenation and translation of the names of document elements.
% https://www.overleaf.com/learn/latex/Spanish
% \usepackage{hyphenat} % Paquete para controlar la separación de palabras
% \hyphenation{mate-máti-cas recu-perar}



% ################################################################
% #######              GEOMETRY                     ##############
% ################################################################

% ************************* Geometría ********************************
% \usepackage[a4paper, inner=1.5cm, outer=3cm, top=4cm, bottom=3cm, bindingoffset=1cm]{geometry}
% \usepackage[a4paper, left=37mm,right=30mm,top=35mm,bottom=30mm]{geometry}
% \usepackage[a4paper, left=2.0cm, right=2.0cm, top=3.0cm, bottom=3.0cm]{geometry}
\usepackage[a4paper, inner=2.0cm, outer=3.5cm, top=4.0cm, bottom=3.0cm, marginpar=2.0cm, headheight=14.5pt]{geometry}

% use option "showframe" in geometry to see frames
% \usepackage[showframe]{geometry}

% ************************* Layout visualization ********************************

\usepackage{layout} % To see the layout of the page
% \usepackage{showframe}


% ################################################################
% #######     ESTILOS EN GENERAL                    ##############
% ################################################################

% *********************** Cabeceras ******************************
\usepackage{fancyhdr} % Permite cambiar el estilo de las cabeceras y pies de página de las páginas del documento
% https://ctan.javinator9889.com/macros/latex/contrib/fancyhdr/fancyhdr.pdf

% \usepackage{fncychap} % Este paquete permite cambiar el estilo de la cabecera de los capítulos
\usepackage[sf,outermarks]{titlesec} % Este paquete permite cambiar el estilo de las cabeceras, pies de página y la forma de las secciones y divisores de sección.
% \usepackage[compact]{titlesec}
% \usepackage{sectsty}


% ************************* ToC and Index ********************************
\usepackage[nottoc]{tocbibind} % Para añadir las listas de figuras y cuadros a la TOC
% \usepackage[acronym,footnote,nonumberlist,toc]{glossaries}
% Erabilera
% http://en.wikibooks.org/wiki/LaTeX/Glossary
% latexmk erabiliz gero, ikusi http://tex.stackexchange.com/questions/1226/how-to-make-latexmk-use-makeglossaries

% Glosario-en eskuliburu zabaldua
% http://osl.ugr.es/CTAN/macros/latex/contrib/glossaries/glossaries-user.html#x1-140002.2
\usepackage[page]{appendix} % Añade el enviroment appendices
% \usepackage[toc,page]{appendix} % Añade una tabla de apéndices a la ToC
% https://osl.ugr.es/CTAN/macros/latex/contrib/appendix/appendix.pdf
% \addto{\captionsspanish}{
%     \renewcommand*{\appendixpagename}{Ap\'{e}ndices}
%     \renewcommand*{\appendixtocname}{Ap\'{e}ndices}
% }
\usepackage{makeidx} % Para crear el index si existe

% *********************** Global paragraph indentation ******************************
\usepackage{parskip} % El único propósito es quitar el indent de todos los párrafos
\usepackage{setspace} % spacing environment
%\usepackage[protrusion=true,expansion=true]{microtype}



% ################################################################
% #######     COLOURS                               ##############
% ################################################################

% *************************** Colours *****************************

\usepackage{color}
\usepackage[table, svgnames, pdftex,dvipsnames]{xcolor} %  Coloured text etc.
% https://ctan.javinator9889.com/macros/latex/contrib/xcolor/xcolor.pdf
\usepackage{colortbl}



% ################################################################
% #######     GRAPHICS                              ##############
% ################################################################

% *************************** Graphics and figures *****************************
\usepackage{epsf, graphics, graphicx}

\usepackage[figuresright]{rotating}
%\usepackage{wrapfig}

\usepackage{float} % Fuerza las figuras a una posición exacta

\usepackage[font=small,labelfont=bf]{caption}
\usepackage{subcaption} % Permite poner subsubtítulos dentro de figuras compuestas
% http://mirrors.ibiblio.org/CTAN/macros/latex/contrib/caption/subcaption.pdf

% \usepackage{tikz} % Advanced inline graphics.
% % https://www.bu.edu/math/files/2013/08/tikzpgfmanual.pdf
% % https://ctan.javinator9889.com/graphics/pgf/base/doc/pgfmanual.pdf
% \usepackage{pgfplots}
% \usetikzlibrary{shapes, decorations, calc, arrows}
% \usetikzlibrary{3d,fit,backgrounds, decorations.text}
% \usetikzlibrary{positioning, shapes.symbols}
% \usetikzlibrary{decorations.pathreplacing, calligraphy}
% \tikzset{>=latex}



% ################################################################
% #######     MATHEMATICS                           ##############
% ################################################################

% ************************* Mathematics ********************************
\usepackage{amsmath}
\usepackage{amssymb}
% \usepackage{amsthm} % Use ntheorem better
\usepackage{ntheorem} % Customizing and writing theorems
\usepackage{bm} % Bold math symbols
\usepackage{amscd}
\usepackage{latexsym} % Even more operators. http://mirrors.ibiblio.org/CTAN/macros/latex/base/latexsym.pdf
\usepackage{calc} % Para hacer cálculos de longitudes en los comandos

% *********************************** SI Units *********************************
\usepackage{siunitx} % use this package module for SI units


% ************************* Text typesetting extras ********************************
\usepackage{csquotes} % Para permitir citas en el texto (bonitas)
\usepackage{circledsteps} % Permite crear texto y números encerrados en círculos
% \usepackage[activate=true, final, tracking=true, kerning=true, spacing=true, factor=1100, stretch=10, shrink=10]{microtype}
% \usepackage{microtype}

\usepackage{lipsum}                     % Dummytext
\usepackage{xargs}                      % Use more than one optional parameter in a new commands

% ################################################################
% #######     INSERTABLES                           ##############
% ################################################################

% ************************* Beatiful colorboxes (breakable) ********************************
% \usepackage{mdframed}
\usepackage[most]{tcolorbox} % https://osl.ugr.es/CTAN/macros/latex/contrib/tcolorbox/tcolorbox.pdf

% ******************************* Itemize and enumerate *********************************

\usepackage{paralist} % compactenum...
\usepackage{enumitem} % Customizable enums
% https://ctan.javinator9889.com/macros/latex/contrib/enumitem/enumitem.pdf

% ******************************* Tables *********************************

% https://ctan.javinator9889.com/macros/latex/contrib/booktabs/booktabs.pdf
\usepackage{array} % For defining special column types
\usepackage{ragged2e} % For new types of columns
\usepackage{longtable} % For breakable tables
\usepackage{multirow} % Table cells that span multiple rows
\usepackage{multicol} % Table cells that span multiple columns
% https://osl.ugr.es/CTAN/macros/latex/required/tools/array.pdf
\usepackage{tabulary}
% https://osl.ugr.es/CTAN/macros/latex/contrib/tabulary/tabulary.pdf
% \usepackage{xltabular}
% https://www.ctan.org/pkg/xltabular
%\usepackage{tabularx}
% https://ctan.org/pkg/tabularx?lang=en
\usepackage{booktabs} % For professional looking tables

% *************************** Listings *****************************

% http://tug.ctan.org/tex-archive/macros/latex/contrib/caption/caption-eng.pdf
% \usepackage{listingsutf8} % - Para escribir código
\usepackage{listings}
% https://www.overleaf.com/learn/latex/Code_listing
% https://osl.ugr.es/CTAN/macros/latex/contrib/listings/listings.pdf

% \usepackage{minted} % Substitute for listings
% https://github.com/gpoore/minted/blob/master/source/minted.pdf
% \usepackage{textcomp} % XML kodea formateatzeko

% ************************* Margin Notes ********************************
\usepackage{marginnote}
\usepackage{marginfix}
% \usepackage{mparhack}

% *********************************** TODO Comments *********************************
% \setlength {\marginparwidth }{2cm}
\usepackage[colorinlistoftodos,prependcaption,textsize=scriptsize,textwidth=2cm]{todonotes}


% ################################################################
% #######     REFERENCES AND BIBLIOGRAPHY           ##############
% ################################################################

% ************************* Bibliography ********************************
\usepackage[backend=biber,style=ieee,autocite=plain,sorting=ynt]{biblatex}

% ************************* Referencias y enlaces ********************************
\usepackage{url} % https://osl.ugr.es/CTAN/macros/latex/contrib/url/url.pdf

\usepackage[hyperindex,bookmarks,colorlinks=true,citecolor=blue,urlcolor=blue,linkcolor=blue,pdftex,unicode,breaklinks]{hyperref} % Se recomienda que sea el último paquete en ser importado

\usepackage{cleveref} % Clever references (éste debe ser en realidad el último paquete importado)

% line in order to check if utf-8 is properly configured: áéíóúñ