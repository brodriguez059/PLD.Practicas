\chapter{Control de versiones}\label{chap:design-history}

% \change[inline]{A design document is a continuously changing reference tool. Most of your
% teammates won't have time to read the whole document over and over again every
% time that a new version is released, so it is good to alert them to any
% significant modifications or updates that you have made. As you can see, each
% version will have its own section where you list the major changes made in that
% iteration. If you use a wiki, this section will be replaced by the editing
% history feature of the software. This makes it simple and effortless to track
% changes to the document and to backtrack changes if it becomes necessary.}

% P requiere del paquete ragged2e

\begin{longtable}[H]{
    @{}
    >{\Centering}m{\dimexpr0.12\textwidth-1\tabcolsep\relax}
    M{\dimexpr0.88\textwidth-1\tabcolsep\relax}
    @{}
    }%

    \toprule        % table caption, ref label
    \textbf{Versión} & \multicolumn{1}{c}{\textbf{Descripción}}\\      % head first part of table
    \midrule        % line head body
    \endfirsthead   % Definition of 1. table header

    \multicolumn{2}{c}{Continuación de la página anterior}\\
    \toprule
    \textbf{Versión} & \multicolumn{1}{c}{\textbf{Descripción}}\\      % head following parts of table
    \midrule        % line head body
    \endhead        % Delongtab1finition of all following headers

    \midrule
    \multicolumn{2}{c}{Continua en la siguiente página}\\ % footer 1. (and more) part(s) of table
    \endfoot        % foots of the table without the last one

    \bottomrule
    \caption{Historial de cambios del informe \label{tab:design-history}} \\
    \endlastfoot    % the last(!!) foot of the table

    v0.0 & Iniciado el documento y generadas las secciones y capítulos    \\
    v0.1 & Especificado el público objetivo                               \\
    v0.2 & Redactado el primer game logline                               \\
    v0.3 & Cambios en la maquetación del informe                          \\
    v0.4 & Especificadas las plataformas del juego                        \\
    v0.5 & Generada la sinopsis del gameplay                              \\
    v0.6 & Redactado el apartado de competidores                          \\
    v0.7 & Cambios en la maquetación del informe                          \\
    v0.8 & Redactada la perspectiva/vista del jugador                     \\
    v0.9 & Añadida la lista de modos del juego                            \\
    v1.0 & Añadido el apartado de recursos del gameplay                   \\
    v1.1 & Añadidos apartado sobre descripción de entidades del gameplay  \\
    v1.2 & Añadido el apartado de acciones del gameplay                   \\
    v1.3 & Añadidas las interfaces de los modos sin sus descripciones     \\
    v1.4 & Añadidas condiciones básicas de victoria y puntuación          \\
    v1.5 & Añadido capítulo de referencias                                \\
    v1.6 & Añadida la explicación de los elementos de las interfaces      \\
    v1.7 & Añadida información sobre personajes y entidades               \\
    v1.8 & Terminada la redacción de los modos de juego                   \\
    v1.9 & Expandida la redacción de la descripción de entidades          \\
    v2.0 & Añadida información sobre historia del juego                   \\
    v2.1 & Expandida y finalizada la definición básica del estilo visual del
    juego \\
    v2.2 & Finalizada la definición de los personajes jugables\\
    % v2.3 & \\
    % v2.4 & \\
    % v2.5 & \\
    % v2.6 & \\
\end{longtable}