\subsection{Interfaces} %     5.3.1 Interfaces

\change[inline]{Create wireframes, as described on page 439, for every interface
the artists will need to create. Each wireframe should include a description of
how each interface feature functions. Make sure you detail out the various
states for each interface.}

\urgent{NO OLVIDES AÑADIR LA DESCRIPCIÓN DE CADA ELEMENTO DE CADA INTERFAZ}

\subsubsection{Modo menú principal}
\begin{figure}[H]
    \centering
    \includegraphics[width=0.6\textwidth]{5-Cuerpo/Chapter5/I1.png} %
    \caption{Interfaz del menú principal}
    \label{fig:Interface_Menu_Principal}
\end{figure}
\begin{enumerate}\setcounter{enumi}{-1}
    \item \textbf{Nombre del juego}: Se mostrará el título del juego en la
    pantalla de inicio.
    \item \textbf{Botón de inicio}: Se usará este botón para pasar al menú de
    selección de perfil y así iniciar el juego como tal.
    \item \textbf{Botón de configuración}: Se usará este botón para pasar al
    menú de configuración general.
    \item \textbf{Botón de salida}: Se usará este botón para terminar con la
    ejecución del juego y salir así de éste.
\end{enumerate}

\subsubsection{Modo selección de perfil}
\begin{figure}[H]
    \centering
    \includegraphics[width=0.6\textwidth]{5-Cuerpo/Chapter5/I2.png} %
    \caption{Interfaz del menú de selección de perfiles}
    \label{fig:Interface_Seleccion_Perfil}
\end{figure}
\begin{enumerate}\setcounter{enumi}{3}
    \item \textbf{Nombre de la interfaz}: Se muestra el nombre de la interfaz y
    modo actual.
    \item \textbf{Información de perfil}: Se mostrará el nombre y los datos de
    aquellos perfiles que ya hayan sido creados.
    \item \textbf{Botón de regreso}: Este botón permitirá volver a la interfaz
    anterior a la actual; o mejor dicho, a la interfaz desde la cual se accedió
    a la actual.
    \item \textbf{Eliminar perfil}: Se dará la posibilidad de eliminar los datos
    de un perfil ya creado.
    \item \textbf{Espacio de perfil}: La interfaz mostrará también los espacios
    de perfil que todavía no se estén usando para poder permitir crear uno nuevo
    estos.
    \item \textbf{Crear perfil}: Este botón permitirá la creación de un nuevo
    perfil dentro de un espacio no ocupado.
    \item \textbf{Botón de configuración}: Este botón llevará a la interfaz de
    configuración general del juego.
\end{enumerate}

\subsubsection{Modo perfil}
\begin{figure}[H]
    \centering
    \includegraphics[width=0.6\textwidth]{5-Cuerpo/Chapter5/I3.png} %
    \caption{Interfaz de visualización de perfil}
    \label{fig:Interface_Perfil}
\end{figure}
\begin{enumerate}\setcounter{enumi}{10}
    \item \textbf{Bloque de datos generales del perfil}: En este espacio se
    mostrarán los datos generales del perfil como lo es el nombre del jugador.
    \item \textbf{Bloque de estadísticas del perfil}: En este espacio se
    mostrarán las estadísticas asociadas al perfil como tal.
    \item \textbf{Botón de inicio del juego}: Para poder empezar una nueva
    partida, se pulsará este botón con el fin de ir al menú de configuración de
    partida.
\end{enumerate}

\subsubsection{Modo configuración de partida}
\begin{figure}[H]
    \centering
    \includegraphics[width=0.6\textwidth]{5-Cuerpo/Chapter5/I4.png} %
    \caption{Interfaz del menú de configuración de partida}
    \label{fig:Interface_Configuracion_Partida}
\end{figure}
\begin{enumerate}\setcounter{enumi}{13}
    \item \textbf{Selector de modo}: Este selector permitirá especificar si se
    desea jugar una partida contra bots (singleplayer) o contra otros jugadores
    (multiplayer).
    \item \textbf{Selector de dificultad}: Este selector permitirá especificar
    la dificultad con la que se desea jugar.
    \item \textbf{Botón de creación de partida}: Una vez configurada la partida
    o simplemente dejado los valores por defecto de la configuración, este botón
    permitirá acceder al menú de selección de personaje y crear la partida.
\end{enumerate}

\subsubsection{Modo selección de personaje}
\begin{figure}[H]
    \centering
    \includegraphics[width=0.6\textwidth]{5-Cuerpo/Chapter5/I5.png} %
    \caption{Interfaz del menú de selección de personaje}
    \label{fig:Interface_Seleccion_Personaje}
\end{figure}
\begin{enumerate}\setcounter{enumi}{16}
    \item \textbf{Vista del personaje}:
    \item \textbf{Botones de cambio de personaje}:
    \item \textbf{Clase del personaje}:
    \item \textbf{Habilidad del personaje}:
    \item \textbf{Información adicional sobre el personaje}:
    \item \textbf{Botón de selección}:
\end{enumerate}

\subsubsection{Modo partida}
\begin{figure}[H]
    \centering
    \includegraphics[width=0.6\textwidth]{5-Cuerpo/Chapter5/I6.png} %
    \caption{Interfaz de la partida}
    \label{fig:Interface_Partida}
\end{figure}
\begin{enumerate}\setcounter{enumi}{22}
    \item \textbf{Barra de vida}:
    \item \textbf{Barra de mana}:
    \item \textbf{Puerta de entrada}:
    \item \textbf{Objeto mágico}:
    \item \textbf{Botón de pausa}:
    \item \textbf{Enemigo}:
    \item \textbf{Habilidad especial}:
    \item \textbf{Personaje}:
\end{enumerate}

\subsubsection{Modo pausa}
\begin{figure}[H]
    \centering
    \includegraphics[width=0.6\textwidth]{5-Cuerpo/Chapter5/I7.png} %
    \caption{Interfaz del menú de pausa}
    \label{fig:Interface_Pausa}
\end{figure}
\begin{enumerate}\setcounter{enumi}{30}
    \item \textbf{Nombre del menú}:
    \item \textbf{Botón continuar}:
    \item \textbf{Botón de configuración}:
    \item \textbf{Botón abandonar}:
\end{enumerate}

\subsubsection{Modo victoria}
\begin{figure}[H]
    \centering
    \includegraphics[width=0.6\textwidth]{5-Cuerpo/Chapter5/I8.png} %
    \caption{Interfaz del menú de victoria}
    \label{fig:Interface_Victoria}
\end{figure}
\begin{enumerate}\setcounter{enumi}{34}
    \item \textbf{Botón de salida}:
\end{enumerate}

\subsubsection{Modo derrota}
\begin{figure}[H]
    \centering
    \includegraphics[width=0.4\textwidth]{5-Cuerpo/Chapter5/I9.png} %
    \caption{Interfaz del menú de derrota}
    \label{fig:Interface_Derrota}
\end{figure}


\subsubsection{Modo configuración general}
\begin{figure}[H]
    \centering
    \includegraphics[width=0.6\textwidth]{5-Cuerpo/Chapter5/I10.png} %
    \caption{Interfaz del menú de configuración general}
    \label{fig:Interface_Configuracion_General}
\end{figure}
\begin{enumerate}\setcounter{enumi}{35}
    \item \textbf{Nombre de sección de configuración}:
    \item \textbf{Volumen de la música}:
    \item \textbf{Volumen de los sonidos}:
    \item \textbf{Barras de control de volumen}:
\end{enumerate}