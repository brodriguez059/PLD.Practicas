\chapter{Gameplay}
\section{Overview} % 5.1 Overview
This is where you describe the core game-play. This should tie directly into your physical or software prototype. Use your prototype as the model, and give an overview of how it functions.

\section{Gameplay description} % 5.2 Gameplay description
Provide a detailed description of how the game functions.

\section{Controls} % 5.3 Controls
Map out the game procedures and controls. Use visualizations like control tables and flowcharts, along with descriptions.
\subsection{Interfaces} %     5.3.1 Interfaces
Create wireframes, as described on page 439, for every interface the artists will need to create. Each wire-frame should include a description of how each interface feature functions. Make sure you detail out the various states for each interface.
\subsection{Rules} %     5.3.2 Rules
If you have created a prototype, describing the rules of your game will be much easier. You will need to define all the game objects, concepts, their behaviors, and how they relate to one another in this section.
\subsection{Scoring/winning conditions}%     5.3.3 Scoring/winning conditions
Describe the scoring system and win conditions. These might be different for single player versus multiplayer or if you have several modes of competition.

\section{Modes and other features} % 5.4 Modes and other features
If your game has different modes of play, such as single and multiplayer modes, or other features that will affect the implementation of the gameplay, you will need to describe them here.

\section{Levels} % 5.5 Levels
The designs for each level should be laid out here. The more detailed the better.

\section{Flowchart} % 5.6 Flowchart
Create a flowchart showing all the areas and screens that will need to be created.

\section{Editor} % 5.7 Editor
If your game will require the creation of a proprietary level editor, describe the necessary features of the editor and any details on its functionality.
\subsection{Features}%     5.7.1 Features
\subsection{Details}%     5.7.2 Details

