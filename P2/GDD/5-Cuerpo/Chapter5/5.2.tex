\section{Descripción del gameplay} % 5.2 Gameplay description

% Aquí debes poner más cosas

\change[inline]{Provide a detailed description of how the game functions.}

% En qué mundo/espacio se va a desarrolar el juego
% Vista del jugador como párrafo nada más

\subsection{Modelo de interacción}

% ¿Cuál es tu modelo de interacción?

% El jugador es un actor interno que interactua con el entorno a través de un personaje
% El jugador tiene un perfil y selecciona un personaje cuando desea jugar una partida
% El jugador puede mover el personaje dentro de una partida
% El jugador puede atacar con el personaje dentro de una partida
% El jugador puede usar las habilidade especiales asociadas al personaje
% El jugador puede recolectar objetos especiales usando el personaje
% El jugador puede hacer uso de botones para usar objetos mágicos
% El jugador puede hacer uso de un botón de menú

\subsection{Objetivos del jugador}

% Completar el nivel (llegar a la cima de la torre)
% Derrotar el mayor número de enemigos (especial para puntuación)
% Recolectar objetos especiales para fastidiar a los compañeros
% Completar el nivel antes que los demás jugadores

\subsection{Retos del jugador}

% Evitar que se pierda toda la vida
% Evitar que se agote el tiempo para llegar a la cima de la torre
% Evitar que los otros jugadores fastidien su progreso
% Evitar quedarse atascado en las plataformas
% Planificar cuándo usar los objetos mágicos
% Planificar cuándo usar las habilidades especiales