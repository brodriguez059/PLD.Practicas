\section{Descripción del gameplay} % 5.2 Gameplay description

% \change[inline]{Provide a detailed description of how the game functions.}

% Aquí debes poner más cosas
% \urgent[inline]{TERMINAR TERMINAR TERMINAR}
% \urgent[inline]{TERMINAR TERMINAR TERMINAR}
% \urgent[inline]{TERMINAR TERMINAR TERMINAR}
% \urgent{NO OLVIDES REDACTAR ESTE APARTADO. NO TE BASTA CON LA VISTA DEL JUGADOR}
% \urgent[inline]{Para poder terminar este apartado primero se necesita tener un prototipo de juego hecho}

% En qué mundo/espacio se va a desarrolar el juego
% Vista del jugador como párrafo nada más
Al ser un juego 2D, la vista a la cual tendrá acceso al jugador será aquella que
muestre una cámara que sigue al personaje seleccionado por éste en la
partida. Por lo tanto, podemos decir que el jugador será capaz de ver un área
alrededor del personaje y que a medida que éste se mueve, la cámara también lo
seguirá.

\subsection{Modelo de interacción}

Consideraremos como “modelo de interacción” al conjunto de acciones que puede
realizar el jugador con el fin de interactuar con el juego. La lista de estas
acciones para este proyecto sería:

\begin{itemize}
    \item El jugador es un actor que interactúa con el entorno de una partida a
    través de un personaje.
    \item El jugador tiene un perfil asignado con el cual puede empezar una
    partida y seleccionar un personaje para jugar.
    \item El jugador puede mover el personaje dentro de una partida.
    \item El jugador puede colisionar con las plataformas de una partida.
    \item El jugador puede atacar a los enemigos de la partida a través del
    personaje.
    \item El jugador puede usar las habilidades especiales asociadas al
    personaje.
    \item El jugador puede recolectar objetos especiales usando el personaje.
    \item El jugador puede hacer uso de botones para usar objetos mágicos.
    \item El jugador puede hacer uso de un botón de menú para configurar el
    juego.
\end{itemize}

\subsection{Objetivos del jugador}
El objetivo principal del jugador dentro de una partida es conseguir llegar a la
cima de la torre antes que nadie, asegurándose de destruir a los enemigos y
jefes necesarios para ello con el fin de liberarse de la maldición del rey mago y escapar.

Sin embargo, podemos listar otros subobjetivos adicionales:
\begin{itemize}
    \item Derrotar el mayor número de enemigos y jefes.
    \item Recolectar objetos especiales para fastidiar el progreso de los demás
    jugadores.
    \item Encontrar todos los secretos y objetos del juego.
\end{itemize}

\subsection{Retos del jugador}

Los retos a los que tiene que someterse el jugador están todos relacionados con
los objetivos del juego:
\begin{itemize}
    \item Evitar perder toda la vida durante la partida
    \item Evitar que los demás jugadores fastidien su progreso con los objetos
    mágicos que han recolectado y usado
    \item Evitar quedarse atascado en las plataformas del nivel
    \item Planificar cuándo y cómo moverse por las plataformas del nivel para
    avanzar rápidamente
    \item Evitar que se agote el tiempo que tiene para llegar a la cima de la
    torre
    \item Planificar cuándo usar los objetos mágicos recolectados
    \item Planificar cuándo usar las habilidades especiales del personaje
    \item Planificar cuándo y cómo usar el dinero ganado en la partida
\end{itemize}