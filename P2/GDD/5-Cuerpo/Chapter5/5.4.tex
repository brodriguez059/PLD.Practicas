\section{Modos y otras funcionalidades} % 5.4 Modes and other features

% \change[inline]{If your game has different modes of play, such as single and
% multiplayer modes, or other features that will affect the implementation of the
% gameplay, you will need to describe them here.}

\subsection{Modos}

\subsubsection{Modos de juego}
Los modos de juego del proyecto se definen en función del tipo de entidad contra la cual quiere competir el jugador. Se tienen entonces dos:

\begin{itemize}
    \item \textbf{Singleplayer}: El modo de juego de un sólo jugador será
    idéntico al modo de juego multijugador, pero los contrincantes de la partida
    serán todos bots. Además, es en este modo que se puede encontrar el final
    secreto del juego.
    \item \textbf{Multiplayer}: El modo de juego multijugador contendrá el
    gameplay descrito hasta ahora. Se competirá contra otros jugadores, se
    utilizarán los objetos mágicos y las habilidades para fastidiar a los
    contrincantes, etc.
\end{itemize}

\subsubsection{Grupos de interacción}

A la hora de implementar, sin embargo, se tendrá una serie de modos o grupos de interacción que definirán el flujo de eventos de la aplicación de cara a empezar la partida. Estos modos están definidos según la principal funcionalidad que cumplen y la lista de éstos es:

\begin{itemize}
    \item \textbf{Modo menú principal}: Al iniciar el juego se mostrará la
    pantalla de inicio que contendrá el menú principal y desde la cual se podrá
    acceder al selector de perfil.

    \item \textbf{Modo selección de perfil}: Dentro del menú de selección de
    perfil, el jugador podrá escoger su perfil para empezar una partida,
    eliminar un perfil ya creado o añadir uno nuevo. Desde este modo se puede
    acceder de nuevo al menú principal y al menú de configuración de perfil.

    \item \textbf{Modo perfil}: Este modo le permitirá al jugador ver
    información de su perfil, configurar elementos como el nombre del perfil o
    acceder al menú de configuración de nueva partida.

    \item \textbf{Modo configuración de partida}: Al seleccionar el
    perfil, el jugador podrá escoger empezar una nueva partida en la cual jugará
    contra bots u contra otros jugadores según su preferencia. Además, podrá
    configurar la información de la partida desde aquí antes de empezar. Se
    podrá acceder al menú de selección de personaje o volver al modo perfil.
    % Singleplayer: Obligatorio para poder jugar con bots
    % Multiplayer: Para jugar con más personas

    \item \textbf{Modo selección de personaje}: Después de crear una partida,
    pero antes de empezarla, el jugador deberá escoger un personaje con el cual
    jugar. Desde este modo se podrán ver los distintos personajes, sus
    habilidades y otras características. Además, desde éste se podrá acceder al
    modo de partida o volver al modo de configuración de partida.

    \item \textbf{Modo partida}: El modo principal del juego en el cual ocurrirá
    todo el gameplay como tal. En éste el jugador podrá controlar su personaje,
    utilizar objetos, derrotar enemigos, competir contra otros jugadores, entre
    otros. Desde éste se puede acceder al modo victoria, modo derrota y modo
    pausa.

    \item \textbf{Modo pausa}: Para que el jugador pueda abandonar una partida o
    entrar al modo de configuración general, deberá primero entrar al modo
    pausa. A pesar de su nombre, este modo no congelará la partida, simplemente
    le permitirá al jugador acceder a las funcionalidades mencionadas
    anteriormente. Desde él se puede abandonar la partida (haciendo que el juego
    considere la partida como perdida) o entrar en el modo de configuración
    general.

    \item \textbf{Modo victoria}: Si el jugador cumple las condiciones de
    victoria dentro del modo partida, se le llevará al modo de victoria. Desde
    éste podrá volver al modo configuración de perfil.

    \item \textbf{Modo derrota}: Si el jugador no cumple con las condiciones de
    victoria, se le llevará al modo derrota. Desde éste podrá volver al modo
    configuración de perfil.

    \item \textbf{Modo configuración general}: También considerado como el
    menú de configuración general, en éste se podrá configurar el volumen de
    los sonidos y la música, además de otras configuraciones. Se podrá a acceder
    a él desde los demás modos.
\end{itemize}

\subsection{Recursos}

\begin{itemize}
    \item \textbf{Monedas}: Durante una partida, el jugador podrá ganar monedas
    ya sea recolectándolas en los niveles o derrotando enemigos y jefes. Éstas
    permitirán comprar objetos mágicos a los distintos vendedores que se podrá
    encontrar en el juego, pero dicho recurso será independiente entre partidas
    y se reiniciará cada vez que se inicie una nueva.
    \item \textbf{Vida}: El personaje del jugador tendrá una vida asignada que
    deberá procurar no perder completamente. Una vez que la vida se reduzca a
    cero, el jugador perderá la partida.
    \item \textbf{Maná}: El personaje del jugador tendrá una cantidad de maná
    asignado que se recuperará con el tiempo. Este recurso será usado para
    activar las habilidades del personaje seleccionado.
\end{itemize}