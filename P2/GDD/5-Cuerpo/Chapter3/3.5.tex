\section{Comparación de características} % 3.5 Feature comparison

% \urgent[inline]{Compare your game to the competition. Why would a consumer
% purchase your game over the others?}

\subsection{Ultimate Chicken Horse}

La similitud que tiene el producto con este juego recae en el hecho de que ambos
son de los géneros \emph{festivo} y \emph{plataformas}, se fomenta la
competición entre jugadores y ambos permiten al jugador hacer uso de objetos
especiales para impedir el progreso de los demás competidores.

Sin embargo, podemos observar las siguientes diferencias entre el juego a
desarrollar y \emph{Ultimate Chicken Horse}:
\begin{itemize}
    \item \emph{Ultimate Chicken Horse} permite partidas multijugador local de
    hasta 4 jugadores; mientras que \emph{\izenburua\ } se espera que pueda
    permitir partidas multijugador local de más de 4 personajes.
    \item \emph{Ultimate Chicken Horse} fomenta la competición, pero tiene un
    flujo de juego con cuellos de botella con respecto a la velocidad de
    reacción que se espera de los jugadores. Se espera que \emph{\izenburua\ }
    mantenga un ritmo de juego constamente rápido en todo momento de una
    partida.
    \item \emph{Ultimate Chicken Horse} tiene una lista de niveles cuyos
    entornos individuales se mantienen siempre iguales internamente. Se espera
    que \emph{\izenburua\ } implemente generación procedural para dar más
    variabilidad a los niveles de juego.
    \item Los personajes seleccionables en \emph{Ultimate Chicken Horse} no
    tienen diferencias a nivel de mecánicas de juego o habilidades; mientras que
    se espera que los personajes de \emph{\izenburua\ } sí las tengan.
    \item Las partidas de \emph{Ultimate Chicken Horse} pueden hacerse largas
    debido a los cuellos de botella mencionados anteriormente; mientras que se
    espera que las partidas de \emph{\izenburua\ } sean rápidas y ágiles.
\end{itemize}


\subsection{Jackbox Party Packs}

La similitud que tiene el producto con estos juegos recae en el hecho de que son
todos del género \emph{festivo}, se fomenta la competición entre jugadores y las
partidas suelen ser cortas y rápidas.

Sin embargo, podemos observar las siguientes diferencias entre el juego a
desarrollar y los \emph{Jackbox Party Packs}:
\begin{itemize}
    \item Es necesario comprar distintos juegos/paquetes de \emph{Jackbox Party
    Pack} para ampliar el repertorio de juegos que se pueden experimentar;
    mientras que la compra de \emph{\izenburua\ } será única y contendrá todo el
    contenido en su totalidad.
    \item En los \emph{Jackbox Party Packs} los juegos tienen una fase de
    votación para decidir quién es el ganador; mientras que en \emph{\izenburua\
    } el ganador se decide puramente en base a la habilidad del jugador.
    \item En los \emph{Jackbox Party Packs} no se dispone de objetos especiales
    para progresar o ralentizar el progeso ajeno; mientras que en
    \emph{\izenburua\ } estos serán parte fundamental de cada partida.
    \item Los \emph{Jackbox Party Packs} sólo están disponibles en inglés y han
    sido desarrollados para un público anglosajón, lo cual excluye a gran parte
    del público objetivo de \emph{\izenburua\ }.
    \item Las partidas de \emph{Jackbox Party Packs} son siempre por turnos, lo
    cual hace que la velocidad de reacción no deba ser muy rápida; mientras que
    en \emph{Jackbox Party Packs} se espera una velocidad de reacción rápida en
    todas las partidas.
    \item Muchos minijuegos de los \emph{Jackbox Party Packs} suelen ser
    aburridos para un público cuyo origen cultural no es anglosajón y suele
    haber más cantidad que calidad; mientras que en \emph{\izenburua\ }
    contendrá un humor menos acoplado al origen cultural del público objetivo y
    contendrá modos de juego mejor preparados.
\end{itemize}

\subsection{Pummel Party}

La similitud que tiene el producto con estos juegos recae en el hecho de que
ambos son del género \emph{festivo}, se fomenta la competición entre jugadores y
se hace uso de distintos objetos para progresar en cada partida.

Sin embargo, podemos observar las siguientes diferencias entre el juego a
desarrollar y \emph{Pummel Party}:

\begin{itemize}
    \item En \emph{Pummel Party}, algunos videojuegos sí que contienen elementos
    de acción, pero el modo principal de juego es por turnos, lo cual genera un
    cuello de botella para la velocidad de reacción que se espera del jugador.
    En \emph{\izenburua\ } se espera una velocidad de juego constamente rápida
    cada partida.

    \item Los personajes seleccionables en \emph{Pummel Party} no
    tienen diferencias a nivel de mecánicas de juego o habilidades; mientras que
    se espera que los personajes de \emph{\izenburua\ } sí las tengan.

    \item \emph{Pummel Party} puede llegar a sufrir problemas de balance en
    muchas partidas. Se espera que los elementos y objetos de \emph{\izenburua\
    } sean adecuadamente balanceados.
\end{itemize}
