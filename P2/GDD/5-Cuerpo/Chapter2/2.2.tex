\section{Sinopsis del gameplay}%     2.2 Gameplay synopsis

% \urgent[inline]{Describe how your game plays and what the user experiences. Try
% to keep it concise: no more than a couple of pages. You might want to reference
% some or all of the following topics:}

% \textbf{Uniqueness}: What makes your game unique?

\emph{\izenburua } es un juego festivo de acción y plataformas 2D en el que tú y
tus amigos competís para ser el primero en llegar a la cima de la torre maldita
y destruir al malvado y demente rey mago con el fin de liberarte de la maldición
que éste te ha lanzado.

% It is an action platform game, but you race and compete against your
% teammates. You gotta be quick or be capable of make them slow.

% \textbf{Mechanics}: How does the game function? What is the core play mechanic?

% You use artifacts and objects to disrupt your friends' progress.
% You fight against monsters
% You must be quick at moving throught the levels and platforms

Deberás evitar caer en las manos de los múltiples monstruos, trampas y ataques
de tus contrincantes mientras tú mismo te aseguras de avanzar rápidamente por
los niveles y eradicar el progreso de los demás usando magia olvidada y
artefactos antiguos.

% \textbf{Setting}: What is the setting for your game: the Wild West, the moon, medieval times?

La ambientación estará inspirada en diferentes eras antiguas como lo son el
medievo y la Antigua Grecia, pero aparecerán elementos y conceptos modernos
relacionados al teatro a lo largo del juego. Siempre dejando bien sabido que se
desconoce el momento exacto en el que transcurren los eventos del juego y
generando una sensación de incertidumbre.

% \textbf{Look and feel}: Give a summary of the look and feel of the game.

El juego tendrá un estilo visual \emph{Pixel-Art} relativamente simple para
simplificar el alcance del proyecto.\urgent{SE CONSIDERA QUE IGUAL ES MEJOR ESCOGER UN ESTILO VISUAL DIFERENTE PARA REDUCIR LA DIFICULTAD DEL PROYECTO}