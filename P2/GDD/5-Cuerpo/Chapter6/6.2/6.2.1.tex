\subsection{Personajes}\label{subsec:playable-characters}

% Nombres que tengan que ver con marionetas, actores y actrices manipulad@s
%Una habilidad de movimiento y una de disrupción % Truman, Jim

La descripción de los personajes jugables requerirá explicar una serie de puntos
básicos para acotar correctamente la implementación. Se tiene, por tanto, que
cada personaje deberá definir:
\begin{itemize}
    \item \textbf{Descripción}: Un resumen sencillo del estilo del personaje y cómo
    justificar la temática de sus ataques y habilidades.
    \item \textbf{Ataque básico}: El ataque simple del personaje que se hará al utilizar
    el botón de ataque.
    \item \textbf{Habilidad de movimiento}: Una de las habilidades de las dos habilidades
    del personaje. Esta habilidad le deberá otorgar una ventaja de movilidad al
    personaje. Ejemplos de esto sería: poder desplazarse más rápido, permitir
    llegar a más sitios, evitar más facilmente a los enemigos, entre otros.
    \item \textbf{Habilidad de disrupción}: La otra de las dos habilidades del personaje.
    Esta habilidad le permitirá al personaje interrumpir el progreso de los
    demás contrincantes. Ejemplos de esto sería: aumentar el número de enemigos
    que lo ataque, hacer aparecer más trampas, hacer que alguno de sus recursos
    desaparezca, entre otros.
\end{itemize}

Tomando esto en cuenta, la lista de personajes jugables del proyecto sería:

\begin{itemize}
    % \item Bard
    \item \textbf{Mago}
    \begin{itemize}
        \item \textbf{Descripción}: Así como su nombre indica, el mago es un personaje
        habilidades provenientes de sus intensivos estudios e
        investigaciones arcanas. Sus habilidades y ataques seguirán la misma
        temática mágica.
        \item \textbf{Ataque básico}: El mago lanzará un rayo mágico desde su bastón.
        Debido a que este personaje es maestro de este hechizo, no hará uso de
        maná cuando lo use.
        \item \textbf{Habilidad de movimiento}: El mago es capaz de teletransportarse a
        una corta distancia, esquivando trampas o enemigos.
        \item \textbf{Habilidad de disrupción}: Se forma un portal en el nivel del
        contrincante desde el cual aparecen nuevos enemigos. La cantidad varía
        en cada uso, sacando de 3 a 6 enemigos que irán directos a atacar al
        contrincante.
    \end{itemize}
    \item \textbf{Ladrón}
    \begin{itemize}
        \item \textbf{Descripción}: El ladrón es un personaje con habilidades
        provenientes de su extensa experiencia en el sigilo y la vida citadina
        nocturna. Sus habilidades y ataques seguirán la misma temática por esa
        misma razón.
        \item \textbf{Ataque básico}: El ladrón hará uso de su navaja rápidamente para
        poder realizar múltiples ataques.
        \item \textbf{Habilidad de movimiento}: El ladrón se vuelve invisible durante 5
        segundos, haciendo que los enemigos no puedan atacarle.
        \item \textbf{Habilidad de disrupción}: El ladrón es capaz de robar un objeto
        aleatorio de uno de sus contrincantes.
    \end{itemize}
    \item \textbf{Druida}
    \begin{itemize}
        \item \textbf{Descripción}: El druida es un personaje con habilidades
        provenientes de su extenso conocimiento de la naturaleza y sus múltiples
        años viviendo en el bosque e interactuando con animales y plantas. Sus
        habilidades y ataques seguirán la misma temática.
        \item \textbf{Ataque básico}: El druida lanzará un hechizo de mal de ojo que será
        muy similar al ataque del mago. Una vez impacte con el enemigo
        simplemente le hará daño.
        \item Habilidad de movimiento: El druida es capaz de realizar un doble
        salto al crear una planta debajo de sus pies a modo de soporte que se
        desvanece al instante.
        \item \textbf{Habilidad de disrupción}: El nivel de contrincante se llena de
        enredaderas y lodo, haciendo que su velocidad baje. Dichas trampas no
        afectarán a los enemigos.
    \end{itemize}
    % \item Paladin    %
    % \item Warrior    %
    % \item Sorcerer   %
    \item \textbf{Bárbaro}  %
    \begin{itemize}
        \item \textbf{Descripción}: El bárbaro es un personaje con habilidades
        provenientes de su extenso transfondo de esfuerzo físico y batallas. Sus
        habilidades y ataques seguirán una temática que le darán aires de
        mercenario.
        \item \textbf{Ataque básico}: El bárbaro utilizará su gran espada montante,
        haciendo más daño contundente que cortante.
        \item \textbf{Habilidad de movimiento}: El bárbaro se vuelve rojo de rabia,
        moviéndose rápidamente hacia adelante por unos segundos y haciendo daño
        a los enemigos que estén en su camino.
        \item \textbf{Habilidad de disrupción}: Ruge y grita como una bestia, aumentando
        su ataque base y haciendo que los enemigos del nivel del contrincante se
        enfurezcan por los \textit{insultos} profanados por el bárbaro.
    \end{itemize}
    % \item Artificier %
\end{itemize}