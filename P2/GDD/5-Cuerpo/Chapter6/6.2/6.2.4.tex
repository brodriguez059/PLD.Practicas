\subsection{Vendedores}

% for each:
% \subsubsection{Behaviour} % 6.2.2.1 Behavior
% \subsubsection{AI} % 6.2.2.2 AI

\begin{itemize}
    \item \textbf{Vendedor de comida}: El vendedor de perritos está al tanto
    del verdadero plan del Rey Mago y sabe que todo buen espectáculo debe ser
    acompañado por comida, para sacar ganancia de los hambrientos espectadores
    tenemos a este vendedor. Este vendedor vende comida para recuperar puntos de
    vida o el maná.

    \item \textbf{Vendedor ambulante}: Un vendedor que lleva años perdido en la
    torre, entro para hacer un mapa de ésta y conseguir una fortuna, pero
    terminó perdiéndose en el intento. Sin embargo, esto no quiere decir que haya
    abandonado su profesión. Trata de vender objetos mágicos que se ha
    encontrado por la torre.

    % \item \textbf{Vendedor de armas}: Un herrero que fuera a donde fuera nadie
    % le compraba armas, debido a que era un muy mal herrero y siempre había uno
    % mejor disponible. Sin embargo, en la torre no hay nadie que venda armas, ahí
    % es donde el guerrero vió su oportunidad de negocio. Este herrero te venderá
    % armas de dudosa calidad.
\end{itemize}