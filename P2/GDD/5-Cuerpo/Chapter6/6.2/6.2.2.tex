\subsection{Monstruos}

% for each:
% \subsubsection{Behaviour} % 6.2.2.1 Behavior
% \subsubsection{AI} % 6.2.2.2 AI

La explicación de los monstruos del juego se realizará dividiéndolos por sus
principales tipos de inteligencia artificial para aprovecharse de las
similitudes de comportamiento.

\subsubsection{Atacante volador} % (murcielagos, insectos gigantes, nubes, \dots)

Los enemigos con este tipo de inteligencia artificial no se verá afectados por
la gravedad ya que son capaces de volar. Dejando eso de lado, el comportamiento
que tendrán como entidades hostiles es el de patrullar, detectar al jugador y
atacar.

\begin{itemize}
    \item \textbf{Ojo volador}: Un gran monstruo nacido de un ojo que fue mutado
    por la magia de la torre. Al detectar al jugador, volará en su dirección y
    cada cierto tiempo se lanzará hacia él para atacarle.
    \item \textbf{Nubes mágicas}: La manifestación de la magia de la torre.
    Estas nubes se forman cuando la concentración de magia en un punto es
    excesiva, tomando vida y volviéndose hostil. Al detecar al jugador, volará
    en su dirección y cada cierto tiempo se lanzará hacia él para atacarle.
    Mientras no esté atacando al jugador, la nube será invulnerable a cualquier
    tipo de daño.
\end{itemize}

\subsubsection{Atacante de contacto directo}
% (jabalí, \dots)

Este tipo de entidades, a diferencia de los atacantes voladores, sí se ven
afectados por la gravedad. Dejando eso de lado, el comportamiento esperado de
estos será el mismo: patrullar, detectar y atacar. La diferencia, sin embargo,
recae en el hecho de que los ataques realizados por estos consistirán en que el
cuerpo del propio monstruo haga contacto con el jugador para hacer daño.

\begin{itemize}
    \item \textbf{Animal salvaje}: Un animal que se ha vuelto loco por la
    influencia mágica de la torre. Al detectar al jugador, se moverá en su
    dirección y cargará contra él haciendo daño al colisionar con éste.
\end{itemize}


\subsubsection{Atacante de contacto cercano}
% (soldados, \dots)

El comportamiento de estos monstruos será el mismo que los atacantes de contacto
directo. La diferencia recae en el hecho de que para poder hacer daño, deberán
acercarse al jugador, pero el ataque no consistirá en una colisión cuerpo a
cuerpo con el jugador, sino que en una colisión entre un arma o extensión y el
cuerpo del jugador.

\begin{itemize}
    \item \textbf{Entidad mágica}: Un monstruo nacido de la mutación y mezcla de fluidos y
    materia residual que se ha quedado en la torre con el tiempo. Al detectar al
    jugador, se moverá en su dirección y una vez esté lo suficientemente cerca,
    lo atacará con sus garras.
    \item \textbf{Soldado no-muerto}: Un soldado revivido gracias a la magia de
    la torre. Al detectar al jugador, se moverá en su dirección y una vez esté
    lo suficientemente cerca, lo atacará con su espada.
\end{itemize}

\subsubsection{Atacante de rango}
% (arqueros, magos, \dots)

El comportamiento de estos monstruos será el mismo que los atacantes de contacto
cercano. La diferencia recae en el hecho de que para poder hacer daño,
procurarán alejarse del jugador para poder posicionarse de tal forma que se
pueda realizar un ataque a larga distancia con un proyectil. El daño ocurrirá
cuando dicho proyectil haga contacto con el cuerpo del jugador.

\begin{itemize}
    \item \textbf{Arquero no-muerto}: Un arquero revivido gracias a la magia de
    la torre. Al detectar al jugador desde una distancia considerable, se
    posicionará para atacarlo con su arco. Si el jugador se acerca demasiado,
    intentará alejarse.
    \item \textbf{Mago no-muerto}: Un mago revivido gracias a la magia de
    la torre. Al detectar al jugador desde una distancia considerable, se
    posicionará para atacarlo con su bastón mágico. Si el jugador se acerca
    demasiado, intentará alejarse.
\end{itemize}

% \begin{itemize}
%     \item \textbf{Isquilox}: De las tinieblas de la historia aparece el
%     legendario Isquilox un burro con cabeza de liebre y cuerpo de liebre.
%     \item \textbf{Espectador molesto}: Un espectador al que no le está gustando
%     la obra y decide atacar a los actores.
%     \item \textbf{Rana}: Se dice que el mago tiene el poder de convertir humanos
%     en ranas y que todas las ranas que hay en la torre son aventureros que no
%     han logrado superarla.
%     \item \textbf{Slime}: Creo que no hay ningún juego de rol que no tenga un
%     enemigo de este estilo, por lo que este juego no iba a ser menos.
%     \item \textbf{Ardilla}: Muchas ardillas entran en la torre en busca de
%     bellotas y se pierden en ésta.
%     \item \textbf{Robot}: El mago tiene algo de ingeniero y como hobbie
%     construye robots.
%     \item \textbf{Mago en prácticas}: El mago se aprovecha de los pobres
%     estudiantes de magos que acaban de terminar la carrera y no encuentran
%     trabajo ofreciéndoles trabajo a cambio de "experiencia".
%     \item \textbf{Alpinista}: Tan alta es la torre que vienen incluso alpinistas
%     a escalarla.
%     \item \textbf{Repartidores de comida}: El mago pide comida a domicilio pero
%     nunca llega. ¿Por qué?, debido a la alta torre por supuesto.
%     \item \textbf{Nubes}: Tan alta es la torre y tanta es la magia acumulada en
%     ella que las nubes cobran vida.
% \end{itemize}