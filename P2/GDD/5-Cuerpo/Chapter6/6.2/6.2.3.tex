\subsection{Jefes}

% for each:
% \subsubsection{Behaviour} % 6.2.2.1 Behavior
% \subsubsection{AI} % 6.2.2.2 AI

\subsubsection{El Rey Mago}

El Rey Mago es el dueño y jefe final de la torre. Su verdadero nombre es
\emph{Oxymandias} y como tal es el principal presentador y espectador de todas
las dificultades de los personajes atrapados. El Rey Mago es un no-muerto que ha
alcanzado mantener su consciencia a través de magia arcana, olvidada y poderosa,
perdiendo en el proceso su carne y volviéndose un esqueleto. El primer personaje
que consiga derrotarlo será liberado de su maldición y podrá escapar de la
torre, mientras que el resto deberán quedarse en ésta.

Ataques:
\begin{itemize}
    \item El Rey Mago hace uso de sus extremidades para golpear a todos los
    personajes que colisionen con estas.
    \item El Rey Mago lanza una serie de hechizos mágicos que invocan múltiples
    proyectiles. Dichos proyectiles mágicos seleccionarán a un personaje
    aleatorio e irán dirigidos hacia él.
    \item El Rey Mago lanza una llamarada de fuego arcano desde su boca haciendo
    de área a todo aquel que termine dentro de éste.
\end{itemize}


\subsubsection{El Minotauro}

% Como en todos los laberintos debe haber un Minotauro y este laberinto no iba a
% ser menos. Es fácil describir a un Minotauro todos tenemos la misma idea de este
% animal, cabeza de hombre y cuerpo de vaca. Con las increíbles habilidades de una
% vaca y un hombre, tales como: mugir, pastar, ser ordeñado, mantenerse sobre dos
% patas y por supuesto, mantenerse sobre cuatro patas.

Debido al comportamiento laberíntico de la torre mágica, el Rey Mago contrató a
un minotauro para ser guardián de la puerta de uno de los pisos de la torre. Los
jugadores deberán enfrentarse eventualmente a él para poder avanzar a luchar
contra el jefe final y escapar de la torre.

Ataques:
\begin{itemize}
    \item El minotauro carga contra el jugador y hace mucho daño al colisionar ambos.
    \item El minotauro se acerca al jugador y hace uso de su hacha para atarle a
    distancia cercana.
    \item Si el jugador se aleja demasiado del minotauro, éste invocará una onda
    mágica en dirección del jugador que hará daño al impactar.
    % \item Expulsa tanta leche por sus ubres que el jugador tiene que nadar y su
    % velocidad baja y se envenena.
    % \item Su mugido (¡olé!) es tan estridente y sonoro, que forma ondas visibles
    % en el aire y ataca a distancia.
    % \item Cuando se cansa de estar a dos patas, se pone a cuatro, pero como está
    % obeso, ya que una vaca no pesa poco precisamente, crea un terremoto que hace
    % caer rocas del techo sobre el jugador.
    % \item A veces el Minotauro se cansa y tiene hambre, por lo que se pone a
    % pastar. Pero en la torre no hay hierba, así que ataca al jugador con sus
    % mordiscos intentando robarle comida.
\end{itemize}


\subsubsection{El Viajero}

Un viajero que entró a la torre hace mucho tiempo y que terminó siendo maldecido
y perdiéndose en ésta. No es un jefe por haber sido contratado por el Rey Mago,
sino porque marcó su territorio y el resto de monstruos del piso lo respetan por
eso. Sin embargo, en el fondo sigue estando atemorizado de todo lo que ve.

Ataques:
\begin{itemize}
    \item Como buen viajero, tiene una mochila con todos sus objetos dentro,
    pero ningún arma. Lanza objetos aleatorios al jugador para hacerle daño y forzarlo a
    alejarse.
    \item Llora tanto por estar encerrado, que dispara chorros de agua que
    impactan contra ti.
    \item Lleva tanto tiempo encerrado que en su locura a veces piensa que eres
    un aliado e intenta abrazarte, estrangulándote.
    \item Te teme e intenta alejarse de ti, yendo hasta la pared más cercana,
    pero sin querer aprieta un botón y enciende una trampa.
\end{itemize}


\subsubsection{Las Escaleras}

% Puede que tenga propiedades laberínticas pero sigue siendo una torre y como toda
% buena torre ésta tiene escaleras. ¿Y a quién le gusta subir escaleras? Sin duda,
% entre todos los Bosses éste es el más terrible, no porque el resto de Bosses den
% pena, sino por sus terribles habilidades, tales como: Ser siempre escaleras de
% subida, tener peldaños altos y tener una increíble cantidad de peldaños.

Ataques:
\begin{itemize}
    \item Las escaleras se convierten en una rampa momentáneamente, haciendo que
    el jugador se deslice hasta el suelo, causándole daño y forzándole a subir
    de nuevo a por la puerta.
    \item La barandilla de la escalera se desprende y transforma su material a
    uno más flexible, convirtiéndose en un látigo y dándole golpes al jugador.
    \item Cuando el jugador se acerca a la puerta de salida en el punto más alto
    la escalera, ésta se ríe en su cara y se desplaza para alejarse de la salida
    y hacerle daño al jugador.
\end{itemize}


%     \item \textbf{El guerrero enchufado}: Como pasa con todos los grandes
%     magos siempre hay una gran mujer detrás, este caso no es distinto. Quien
%     corta el bacalao es ella y desafortunadamente tiene un primo tonto que se
%     cree un gran guerrero, como éste no encontraba trabajo la mujer del mago
%     le enchufó para que trabajara en la torre. Principalmente limpia la torre,
%     pero para que este se sienta realizado le han dado el nombre de guerrero y
%     título de Boss. Y sus increíbles habilidades son: dejarse manchas por todos
%     lados, gritar "¡Que no me pises lo fregado!", pisar lo fregado y
%     llorarle a su prima porque los otros monstruos son malos con él.
%     (En vez de espada lleva una fregona)
%     Ataques:
%     \begin{itemize}
%         \item De vez en cuando, ve una mancha en el suelo y friega toda la
%         superficie, barriendo al jugador en el proceso y dejando charcos de agua
%         con las que te puedes resbalar (y la mancha inicial).
%         \item Si el jugador pisa un charco de agua, el guerrero se enfada y le
%         intenta pegar con la fregona mientras grita "¡Que no me pises lo
%         fregado!".
%         \item A veces, cuando le infringes daño, llora y llama a su prima,
%         haciendo que esta se moleste por lo pesado y llorica que es y tira basura
%         al nivel donde está el Boss para que tenga que limpiar. La basura es
%         dañina para el jugador.
%         \item
%     \end{itemize}

%     \item \textbf{La señora del rey mago}: Los buenos teatros de marionetas
%     tienen titiriteros, nuestro teatro tiene a la señora del mago, que como
%     bien todos sabemos es quien mueve los hilos. Con los poderes que tendría
%     un titiritero omnisciente, tales como: manipular a todos los seres de la
%     torre, enchufar a su primo tonto y hacer que su marido se sienta mal por
%     literalmente cualquier cosa que haga.
%     Ataques:
%     \begin{itemize}
%         \item La señora de la torre no se ensucia las manos por un ser insignificante como el jugador, así que llama a una horda de monstruos para que luchen en su lugar.
%         \item A veces su primo le llama para quejarse, molesta, lo convoca al nivel y le obliga a pegar al jugador con su fregona.
%         \item Como es la persona que lleva los pantalones en su relación, hace que su marido lance rayos desde lo alto de la torre contra el jugador y así ahorra su propio maná.
%         \item Para demostrar que no es la señora de la torre solo por su cara bonita, lanza hechizos contra el jugador. (Estos hechizos ocupan mucho en la pantalla por lo grandes y poderosos que son, por lo que son complicados de esquivar).
%     \end{itemize}

%     \item \textbf{El narrador}: Hace falta alguien que cuente las épicas
%     aventuras que suceden en esta torre, alguien que las exagere para que
%     nadie se de cuenta que en realidad es una simple torre, con una vaca, una
%     persona con problemas de vejiga, un nini y una matrimonio aburrido y un
%     bully que entra en la torre a atacar al pobre matrimonio. Sus habilidades
%     son las siguientes: Exagerar una historia, tergiversar la historia para
%     que el malo sea el bueno y el bueno sea el malo, llevarse el crédito por
%     una historia que no ha escrito y meterse donde no le llaman.
%     Ataques:
%     \begin{itemize}
%         \item Ataques relacionados con los Bosses anteriores. Ej.: hace que el guerrero enchufado convierta su fregona en espada y ataque al jugador.
%         \item El narrador hace creer a los espectadores de que el jugador es mala persona, por lo que los espectadores abuchean al jugador mientras le lanzan tomates.
%         \item
%     \end{itemize}