\section{Tipos} % 6.2 Types

% \change[inline]{If your game involves character types, you will need to treat
% each one as an object, defining its properties and functionality.}

Una vez que se ha empezado una partida y se entra al modo de juego principal, se
podrán clasificar los tipos de personajes en dos grandes categorías: personajes
jugables o \emph{PCs} (del inglés \emph{Playable Characters}); y personajes no
jugables o \emph{NPCs} (del inglés \emph{Non-Playable Characters}).

\subsection{PCs} % 6.2.1 PCs (player characters)

Se define como personaje jugable a todos aquellos personajes que pueden ser
controlados por un jugador. La lista de estos personajes se dará en la
subsección \ref{subsec:playable-characters}.

\subsection{NPCs} % 6.2.2 NPCs (nonplayer characters):

Se define como personaje no jugable a todo aquel personaje que no pueda ser
controlado por el jugador en ningún momento y que, por tanto, deberá ser
controlado por el juego a través de inteligencia artificial y algoritmos. Esta
categoría se divide en los siguientes grupos:

\begin{itemize}
    \item \textbf{Enemigo}: Un enemigo es todo aquel personaje cuyo
    comportamiento por defecto para con el jugador es hostil. Dichos personajes
    serán los que intentarán reducir la vida de los personajes jugables a $0$ y,
    como consecuencia, hacer que el jugador pierda la partida. Los enemigos se
    dividen en dos subgrupos:
    \begin{itemize}
        \item \textbf{Jefe}: Un jefe es un tipo de enemigo especial que
        representará un reto mucho mayor para los jugadores al llegar a un
        determinado nivel. Derrotar al jefe será necesario para poder continuar
        a diferencia del otro tipo de enemigo.
        \item \textbf{Monstruo}: Un monstruo es un enemigo que puede ser
        encontrado a lo largo de los niveles del juego mientras se avanza. No es
        necesario derrotarlos para poder avanzar, pero hacerlo recompensará al
        jugador con monedas a coste de poder perder puntos de vida si no se
        tiene cuidado.
    \end{itemize}
    \item \textbf{Vendedor}: Un vendedor es todo aquel personaje cuyo
    comportamiento por defecto para con el jugador es neutro y cuya finalidad es
    la de permitir al jugador hacerse con objetos a cambio de monedas. Podrán
    aparecer en los niveles con una determinada probabilidad.
    \item \textbf{Visualizadores}: Un visualizador es todo aquel personaje cuyo
    comportamiento por defecto para con el jugador es pacífico. Estos personajes
    serán decorativos y no otorgarán recursos, pero servirán como nexo con el
    transfondo e historia del juego.
\end{itemize}