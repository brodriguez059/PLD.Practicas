\section{Tipos} % 6.2 Types

\subsection{PCs (Player Characters)} % 6.2.1 PCs (player characters)

\begin{itemize} % Nombres que tengan que ver con marionetas, actores y actrices manipulad@s
%Una habilidad de movimiento y una de joder
% Truman, Jim
    % \item Bard       %
    \item Mago       %
    \begin{itemize}
        \item Descripción: Un mago celoso del gran mago, debido a la... ¿bella dama que lo acompaña? Para conquistar a la "bella dama", mostrará que es más poderoso que el gran mago derrotándolo.
        \item Habilidad de movimiento: El mago es capaz de teletransportarse a una corta distancia, esquivando trampas o enemigos.
        \item Habilidad de joder: Se forma un portal en el nivel del contrincante donde se spawnean nuevos enemigos. La cantidad varía en cada uso, sacando de 3 a 6 enemigos.
    \end{itemize}
    \item Ladrón      %
    \begin{itemize}
        \item Descripción: Un gran ladrón de guante blanco que deduce que como la torre es alta, esta guarda muchos tesoros.
        \item Habilidad de movimiento: El ladrón se vuelve invisible durante 5 segundos, haciendo que los enemigos no puedan atacarle.
        \item Habilidad de joder: El ladrón es capaz de robar un objeto random de uno de sus contrincantes.
    \end{itemize}
    \item Druida     %
    \begin{itemize}
        \item Descripción: Un druida amante del bosque que está en contra de la torre que han construido en medio del bosque contaminando este.
        \item Habilidad de movimiento: El druida es capaz de realizar doble salto, saltando sobre una rama que crea (animación).
        \item Habilidad de joder: El nivel de contrincante se llena de enredaderas y lodo, haciendo que su velocidad baje. No afecta a los enemigos.
    \end{itemize}
    % \item Paladin    %
    % \item Warrior    %
    % \item Sorcerer   %
    \item Bárbaro  %
    \begin{itemize}
        \item Descripción: Un simple bárbaro al que le gusta destruir cosas y meterse en líos, se mete en la torre en busca de bronca.
        \item Habilidad de movimiento: El bárbaro se vuelve rojo de rabia y hace más daño a sus enemigos.
        \item Habilidad de joder: Ruge y grita como una bestia, haciendo que los enemigos del nivel del contrincante se enfurezcan por los \textit{insultos} profanados por el bárbaro y su ataque aumenta.
    \end{itemize}
    % \item Artificier %
\end{itemize}

\subsection{NPCs (Non-Player Characters)} % 6.2.2 NPCs (nonplayer characters):

\change[inline]{If your game involves character types, you will need to treat
each one as an object, defining its properties and functionality.}

% for each:
% \subsubsection{Behaviour} % 6.2.2.1 Behavior
% \subsubsection{AI} % 6.2.2.2 AI
\begin{itemize}
    \item The Lich King % (The Host) Oxymandias
    \item The monsters/enemies (?)
    \begin{itemize}
        \item Isquilox: De las tinieblas de la historia aparece el legendario Isquilox un burro con cabeza de liebre y cuerpo de liebre.
        \item Espectador molesto: Un espectador al que no le está gustando la obra y decide atacar a los actores.
        \item Rana: Se dice que el mago tiene el poder de convertir humanos en ranas y que todas las ranas que hay en la torre son aventureros que no han logrado superarla.
        \item Slime: Creo que no hay ningún juego de rol que no tenga un enemigo de este estilo, por lo que este juego no iba a ser menos.
        \item Ardilla: Muchas ardillas entran en la torre en busca de bellotas y se pierden en ésta.
        \item Robot: El mago tiene algo de ingeniero y como hobbie construye robots.
        \item Mago en prácticas: El mago se aprovecha de los pobres estudiantes de magos que acaban de terminar la carrera y no encuentran trabajo ofreciéndoles trabajo a cambio de "experiencia".
        \item Alpinista: Tan alta es la torre que vienen incluso alpinistas a escalarla.
        \item Repartidores de comida: El mago pide comida a domicilio pero nunca llega, ¿Por qué? Debido a la alta torre por supuesto.
        \item  Nubes: Tan alta es la torre y tanta es la magia acumulada en ella que las nubes cobran vida.
    \end{itemize}
    \item Bosses
    \begin{itemize}
        \item El Minotauro: Como en todos los laberintos debe haber un Minotauro y este laberinto no iba a ser menos. Es fácil describir a un Minotauro todos tenemos la misma idea de este animal, cabeza de hombre y cuerpo de vaca. Con las increíbles habilidades de una vaca y un hombre, tales como: mugir, pastar, ser ordeñado, mantenerse sobre dos patas y por supuesto, mantenerse sobre cuatro patas.
        
        Ataques:
        \begin{itemize}
            \item Expulsa tanta leche por sus ubres que el jugador tiene que nadar y su velocidad baja y se envenena.
            \item Su mugido (¡olé!) es tan estridente y sonoro, que forma ondas visibles en el aire y ataca a distancia.
            \item Cuando se cansa de estar a dos patas, se pone a cuatro, pero como está obeso, ya que una vaca no pesa poco precisamente, crea un terremoto que hace caer rocas del techo sobre el jugador.
            \item A veces el Minotauro se cansa y tiene hambre, por lo que se pone a pastar. Pero en la torre no hay hierba, así que ataca al jugador con sus mordiscos intentando robarle comida.
        \end{itemize}
        \item El viajero: Un viajero que entro en la torre queriendo ir al baño pero se quedo perdido en esta, debido a sus características laberínticas. No es un Boss debido a su increíble poder, sino porque marco su territorio y el resto de monstruos del piso respetan eso. ¿Como marco su territorio frente a tantos temibles monstruos? La respuesta es simple, no encontró el baño a tiempo. Y como todos los Bosses tiene increíbles habilidades, tales como: no moverse de su esquina por miedo, llorar en su esquina debido a que esta cautivo y por supuesto marcar su territorio.
        
        Ataques:
        \begin{itemize}
            \item Como buen viajero, tiene una mochila con todos sus objetos dentro, pero ningún arma. Tiene tanto miedo de ti que te lanza objetos random para que te alejes.
            \item Llora tanto por estar encerrado, que dispara chorros de agua que impactan contra ti.
            \item Lleva tanto tiempo encerrado que en su locura a veces piensa que eres un aliado e intenta abrazarte, estrangulándote.
            \item Te teme e intenta alejarse de ti, yendo hasta la pared más cercana, pero sin querer aprieta un botón y enciende una trampa.
        \end{itemize}
        \item El guerrero enchufado: Como pasa con todos los grandes magos siempre hay una gran mujer detrás, este caso no es distinto. Quien corta el bacalao es ella y desafortunadamente tiene un primo tonto que se cree un gran guerrero, como éste no encontraba trabajo la mujer del mago le enchufó para que trabajara en la torre. Principalmente limpia la torre, pero para que este se sienta realizado le han dado el nombre de guerrero y título de Boss. Y sus increíbles habilidades son: dejarse manchas por todos lados, gritar "¡Que no me pises lo fregado!", pisar lo fregado y llorarle a su prima porque los otros monstruos son malos con él.
        
        (En vez de espada lleva una fregona)
        
        Ataques:
        \begin{itemize}
            \item De vez en cuando, ve una mancha en el suelo y friega toda la superficie, barriendo al jugador en el proceso y dejando charcos de agua con las que te puedes resbalar (y la mancha inicial).
            \item Si el jugador pisa un charco de agua, el guerrero se enfada y le intenta pegar con la fregona mientras grita "¡Que no me pises lo fregado!".
            \item A veces, cuando le infringes daño, llora y llama a su prima, haciendo que esta se moleste por lo pesado y llorica que es y tira basura al nivel donde está el Boss para que tenga que limpiar. La basura es dañina para el jugador.
            \item 
        \end{itemize}
        \item Las escaleras: Puede que tenga propiedades laberínticas pero sigue siendo una torre y como toda buena torre ésta tiene escaleras. ¿Y a quién le gusta subir escaleras? Sin duda, entre todos los Bosses éste es el más terrible, no porque el resto de Bosses den pena, sino por sus terribles habilidades, tales como: Ser siempre escaleras de subida, tener peldaños altos y tener una increíble cantidad de peldaños.
        
        Ataques:
        \begin{itemize}
            \item La escalera momentáneamente se convierte en una rampa, haciendo que el jugador ruede hasta el suelo, haciéndose daño y teniendo que subir de nuevo.
            \item La barandilla de la escalera se convierte en un látigo.
            \item Cuando estás cerca de la salida (el punto más alto de la escalera), la escalera se ríe en tu cara y se da la vuelta, haciendo que el jugador vuelva al inicio.
            \item 
        \end{itemize}
        \item La señora del rey mago: Los buenos teatros de marionetas tienen titiriteros, nuestro teatro tiene a la señora del mago, que como bien todos sabemos es quien mueve los hilos. Con los poderes que tendría un titiritero omnisciente, tales como: manipular a todos los seres de la torre, enchufar a su primo tonto y hacer que su marido se sienta mal por literalmente cualquier cosa que haga.
        
        Ataques:
        \begin{itemize}
            \item La señora de la torre no se ensucia las manos por un ser insignificante como el jugador, así que llama a una horda de monstruos para que luchen en su lugar.
            \item A veces su primo le llama para quejarse, molesta, lo convoca al nivel y le obliga a pegar al jugador con su fregona.
            \item Como es la persona que lleva los pantalones en su relación, hace que su marido lance rayos desde lo alto de la torre contra el jugador y así ahorra su propio maná.
            \item Para demostrar que no es la señora de la torre solo por su cara bonita, lanza hechizos contra el jugador. (Estos hechizos ocupan mucho en la pantalla por lo grandes y poderosos que son, por lo que son complicados de esquivar).
        \end{itemize}
        \item El narrador: Hace falta alguien que cuente las épicas aventuras que suceden en esta torre, alguien que las exagere para que nadie se de cuenta que en realidad es una simple torre, con una vaca, una persona con problemas de vejiga, un nini y una matrimonio aburrido y un bully que entra en la torre a atacar al pobre matrimonio. Sus habilidades son las siguientes: Exagerar una historia, tergiversar la historia para que el malo sea el bueno y el bueno sea el malo, llevarse el crédito por una historia que no ha escrito y meterse donde no le llaman.
        
        Ataques:
        \begin{itemize}
            \item 
        \end{itemize}
    \end{itemize}
    \item Vendedores
    \begin{itemize}
        \item Vendedor de perritos: Todo buen espectáculo debe ser acompañado por comida, para sacar ganancia de los hambrientos espectadores tenemos a este vendedor. Este vendedor vende comida basura a un precio desorbitado, pero qué vas a hacer si no te dejan entrar con comida.
        \item Vendedor del mapa: Un vendedor que lleva años perdido en el laberinto, entro para hacer un mapa del laberinto y conseguir una fortuna perdiéndose en el intento. Pero eso no quiere decir que haya abandonado su profesión. Trata de venderte cualquier cosa que se encuentre en la torre.
        \item Vendedor de armas: Un herrero que fuera a donde fuera nadie le compraba armas, debido a que era un muy mal herrero y siempre había uno mejor disponible. Sin embargo, en la torre no hay nadie que venda armas, ahí es donde el guerrero vió su oportunidad de negocio. Este herrero te venderá armas de dudosa calidad.
    \end{itemize}
    \item The eyes/cameras (Are these objects?)
    \begin{itemize}
        \item Camarografos: Camarografos que cobran por grabar la épica aventura.
        \item Espectadores: Espectadores de la obra.
    \end{itemize}
\end{itemize}