\subsection{Monstruos}

% for each:
% \subsubsection{Behaviour} % 6.2.2.1 Behavior
% \subsubsection{AI} % 6.2.2.2 AI
\begin{itemize}
    \item \textbf{Isquilox}: De las tinieblas de la historia aparece el legendario Isquilox un burro con cabeza de liebre y cuerpo de liebre.
    \item \textbf{Espectador molesto}: Un espectador al que no le está gustando la obra y decide atacar a los actores.
    \item \textbf{Rana}: Se dice que el mago tiene el poder de convertir humanos en ranas y que todas las ranas que hay en la torre son aventureros que no han logrado superarla.
    \item \textbf{Slime}: Creo que no hay ningún juego de rol que no tenga un enemigo de este estilo, por lo que este juego no iba a ser menos.
    \item \textbf{Ardilla}: Muchas ardillas entran en la torre en busca de bellotas y se pierden en ésta.
    \item \textbf{Robot}: El mago tiene algo de ingeniero y como hobbie construye robots.
    \item \textbf{Mago en prácticas}: El mago se aprovecha de los pobres estudiantes de magos que acaban de terminar la carrera y no encuentran trabajo ofreciéndoles trabajo a cambio de "experiencia".
    \item \textbf{Alpinista}: Tan alta es la torre que vienen incluso alpinistas a escalarla.
    \item \textbf{Repartidores de comida}: El mago pide comida a domicilio pero nunca llega. ¿Por qué?, debido a la alta torre por supuesto.
    \item \textbf{Nubes}: Tan alta es la torre y tanta es la magia acumulada en ella que las nubes cobran vida.
\end{itemize}